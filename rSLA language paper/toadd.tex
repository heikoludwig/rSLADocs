tune the argumentation: introduction should be clear, overview paper, what is the problem in terms of deployment time, reduce the overal time for SLA management, Xlet abstraction, sla language refers to the xlets, easy to use language, run it as a service, fast to deploy, solutiion for sla management and cheap the value proposition, overall motivation

clarify why now: dealing with heterogeneity (wsla applications in application server) notion of service is much more vigor, micro services,  extensibility and customizability, the introduction of devops make infrastructure changes more often than code changes, code agility, 

cloud apps today are being solved by horizontal scaling, once we deploy latency intensive applications SLAs become a must. 

related work why we are doing this, no other comparable solution

introduction -> motivation 

notion of service, metric in the abstract

capture the whole runtime

why now: we are running on shared everything infrastructures, the desire to use SLAs is greater

related work section

specifications, policy language, SNAP : quality of service for computational jobs? 

service now, data warehouse solutions

language model

smart Xlets driven by multiple SLAs

smart Xlet: design 

conceptual model on how we think about metrics, services, SLOs, Xlets

look into the system environment, collecting data from different providers: system scope or system model 
section on highlighting the challenges, 
system model/conceptual elements/language/use case
rSLA language/grammar specification reference 
add doc in the tex repo





@article{czaj,
author={K. Czajkowski and I. Foster and C. Kesselman},
journal={Proc. of the IEEE}, 
title="{A}greement-{B}ased {R}esource {M}anagement",
year={2005}}

@inproceedings{SNAP, 
author = {Czajkowski, Karl and Foster, Ian T. and Kesselman, Carl and Sander, Volker and Tuecke, Steven},
title = {SNAP: A Protocol for Negotiating Service Level Agreements and Coordinating Resource Management in Distributed Systems},
booktitle = {Revised Papers from the 8th International Workshop on Job Scheduling Strategies for Parallel Processing},
series = {JSSPP '02},
year = {2002},
numpages = {31},
publisher = {Springer-Verlag}
}

