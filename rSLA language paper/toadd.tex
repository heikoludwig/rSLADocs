Notes from the discussion yesterday

tune the argumentation: introduction should be clear: overview of paper, 

value proposition, overall motivation

Cloud SLA management can be costly. A provider needs to control resources that are used by multiple tenants. Every tenant agrees on service levels that can be different from those of other customers. To guarantee service compliance, a provider must monitor service level values and schedule measurement and evaluation operations for all active service level agreements. 

Such practices require instrumentation of monitoring, scheduling and data storage tools that can be tedious to deploy for someone unfamiliar with IT. In addition, such practices are performed systematically; hence, a semantic vocabulary is required for their management.

what is the problem in terms of deployment time: the rSLA language reduces the overall deployment time for cloud SLA management 

Xlet abstraction, 
sla language  xlets, easy to use language, run it as a service, fast to deploy, solution for sla management and cheap the 

clarify why now: dealing with heterogeneity (wsla applications in application server) notion of service is much more vigor, micro services,  extensibility and customizability, the introduction of devops make infrastructure changes more often than code changes, code agility, 

cloud apps today are being solved by horizontal scaling, once we deploy latency intensive applications SLAs become a must. 

related work why we are doing this, no other comparable solution

notion of service, metric in the abstract

capture the whole runtime

why now: we are running on shared everything infrastructures, the desire to use SLAs is greater

related work section

specifications, policy language, SNAP : quality of service for computational jobs? 

service now, data warehouse solutions

language model

smart Xlets driven by multiple SLAs

smart Xlet: design 

conceptual model on how we think about metrics, services, SLOs, Xlets

look into the system environment, collecting data from different providers: system scope or system model 
section on highlighting the challenges, 
system model/conceptual elements/language/use case
rSLA language/grammar specification reference 


@article{czaj,
author={K. Czajkowski and I. Foster and C. Kesselman},
journal={Proc. of the IEEE}, 
title="{A}greement-{B}ased {R}esource {M}anagement",
year={2005}}

@inproceedings{SNAP, 
author = {Czajkowski, Karl and Foster, Ian T. and Kesselman, Carl and Sander, Volker and Tuecke, Steven},
title = {SNAP: A Protocol for Negotiating Service Level Agreements and Coordinating Resource Management in Distributed Systems},
booktitle = {Revised Papers from the 8th International Workshop on Job Scheduling Strategies for Parallel Processing},
series = {JSSPP '02},
year = {2002},
numpages = {31},
publisher = {Springer-Verlag}
}

