\section{rSLA editing, runtime}
\label{sec:runtime}
ready to use functions

The rSLA alphabet consists of language elements that require user input for their creation in an rSLA runtime environment. A DSL user can directly 
edit code-block scripts in ruby, start a cloud rSLA service and create an active SLA object. In the diagram solid black arrows indicate that 
user-input is required to create the equivalent objects in an rSLA running environment. 

In Figure \ref{rSLA_diag} language elements that require user-input represent tree branches for the creation of service level agreements. DSL 
user-input requires editing a ruby script to describe the attributes for any new rSLA object. 

The rSLA language also contains elements, whose definition with an rSLA service requires that at least one SLA object exists (ex. base metric, slo). A 
DSL user can edit the context of such elements and associate them with required objects using code blocks in ruby SLA scripts.
