\section{Related work}
rSLA enables rapid cloud deployment of SLA management services. Currently in cloud markets there is no application that provides service level management on-demand, given the availability of a cloud platform.

A significant body of work has explored the problem of automated management of SLAs in dynamic service environments. Two main specifications have been proposed, namely Web Service Level Agreement language (WSLA) from IBM \cite{wsla}, and Web Service Agreement (WS-Agreement) from Open Grid Forum \cite{wsag}. WSLA is primarily designed to capture SLAs for web services and uses the XML standard to facilitate automatic service configuration. WS-Agreement inherits language characteristics of the WSLA specification and has been used in numerous grid or cloud research initiative\cite{soi, butler, cslam}. 




Early industrial research on web service and utility computing \cite{ludwig, IBM1, dan} introduced the notion of on-demand service provisioning, which the cloud computing paradigm later established as the standard provisioning mode for virtual services.

In order to meet the additional challenges posed by the dynamic and complex nature of the cloud environment, the research community has investigated solutions for automated SLA negotiation and scheduling on distributed environments \cite{SNAP, czaj, lessons}, introduced strategies for SLA monitoring \cite{rana} and conceptual schemas for service level aggregations \cite{ulhaq}.
Additionally, the scientific community has looked into solutions for efficient database tenant distribution and workload execution \cite{kraska, nec, sakr}. 

In the industry, however, cloud providers have not yet adopted automated SLA management practices. Instead, they prefer to use third-party services like ServiceNow \cite{servicenow} or Amazon Redshift \cite{redshift} for their service management and data warehousing needs. Such third party services may also provide service performance analytics and tools for administering and reporting on service level violations.

Other research projects that are relevant to the rSLA context include SLAng [4] and RBSLA [5], which is a rule-based SLA that uses RuleML to define and express SLAs. The 


 Two efforts, as far as we know, that appear relevant to our work are the Cloud Service Level Agreement (CSLA) proposed by Kouki et al. [6], and the Service Level Management approach for cloud provisioning developed by ServiceNow [7]. 
 
 Like Kouki et al. and ServiceNow, rSLA offers new approaches to deal with the lack of flexibility support in current SLA management systems, such as dynamic and virtual resource provisioning, QoS guarantee through self-adaptation policies, and automatic and dynamic SLA configuration. Unlike CSLA and the approach offered by ServiceNow, our approach is based on Ruby, which has a native support for flexibility and agility, and thus allows for virtual and flexible instrumentation of SLA services as well as scalability of SLA management.



%3	S. Bajaj, et al. Web services policy framework (WS-Policy), Sept. 2004, http://specs.xmlsoap.org/ws/2004/09/policy/ws-policy0904.pdf
%4	D. D. Lamanna, J. Skene and W. Emmerich, Slang: A language for defining service level agreements, 2003, pp. 100–106.
%5	A. Paschke, RBSLA:A declarative Rule-based Service Level Agreement Language based on RuleML, Proc. of the International Conference on Computational Intelligence for Modelling, Control and Automation, and International Conference on Intelligent Agents, Web Technologies and Internet Commerce (CIMCA-IAWTIC’05), 2005.
%6	Y. Kouki, F. A. de Oliveira, S. Dupont, and T. Ledoux, A Language Support for Cloud Elasticity Management, In 14th IEEE/ACM International Symposium on Cluster, Cloud and Grid Computing (CCGrid), 2014.
%7	http://wiki.servicenow.com/index.php?title=Service_Level_Management_for_Cloud_Provisioning#gsc.tab=0 
