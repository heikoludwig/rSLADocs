\section{Related work}
rSLA enables rapid cloud deployment of SLA management services. Currently in cloud markets there is no application that provides service level management on-demand, given the availability of a cloud platform.

The scientific literature has proposed language specifications \cite{wsla, wsag} for SLAs that are executed over web and/or distributed computing environments. Such SLA specifications have been used in numerous grid or cloud research initiatives \cite{soi, butler, cslam}. Early industrial research on web service and utility computing \cite{ludwig, IBM1, dan} introduced the notion of on-demand service provisioning, which later the cloud computing paradigm established as the standard provisioning mode for virtual services.

Moreover, the research community has investigated solutions for automated SLA negotiation and scheduling on distributed environments \cite{SNAP, czaj, lessons}, introduced strategies for SLA monitoring \cite{rana} and conceptual schemas for service level aggregations \cite{ulhaq}.
Additionally, the scientific community has looked into solutions for efficient database tenant distribution and workload execution \cite{kraska, nec, sakr}. 

In the industry, however, cloud providers have not yet adopted automated SLA management practices. Instead, they prefer to use third-party services like ServiceNow \cite{servicenow} or Amazon Redshift \cite{redshift} for their service management and data warehousing needs. Such third party services may also provide service performance analytics and tools for administering and reporting on service level violations.
