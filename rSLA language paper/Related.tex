\section{Related work}
Since the  QoS of IT services has been measured and managed since a long time a significant body of existing work must be considered. 

Next to the set of typical IT service SLA management products mentioned in the introduction, many providers of Cloud and networking services provide specific service performance information on their Web site or at an API. This is often on the basis of tiered service offerings, with better quality demanding a higher price. While this typically entails some kind of performance guarantee it is not a client specific SLA, tailored to a client's business needs.

An important part of a solution to automate SLA management is a formal, machine-interpretable representation of the SLA. The research community has explored the problem of automated SLA management in dynamic service environments. Several specifications have been proposed in the Web service and Grid context such as the Web Service Level Agreement language (WSLA) \cite{wsla}, the Web Services Offer Language (WSOL) \cite{wsol} and Web Service Agreement (WS-Agreement) standard of the Open Grid Forum \cite{wsag}. 

WSLA is designed to capture SLAs for web services and is used, among other things, to facilitate automatic service configuration, e.g., to configure load balancers of clustered Web servers. WS-Agreement inherits language characteristics of the WSLA specification and, like WSLA, has been used in numerous grid or cloud research initiatives \cite{soi, butler, cslam, kouki}. 

Early industrial research on web service and utility computing \cite{ludwig, IBM1, dan} introduces the notion of on-demand service provisioning, which the cloud computing paradigm later established as the standard provisioning mode for virtual services. QoS has also often be considered in automated service composition \cite{boualem}.

Additional work addresses automated SLA negotiation and scheduling on distributed environments \cite{SNAP, lessons}, introduced strategies for SLA monitoring \cite{rana} and conceptual schemas for service level aggregations \cite{ulhaq}. Additionally, the scientific community has looked into solutions for efficient database tenant distribution and workload execution \cite{kraska, nec, sakr}. 

The specification work enables the automatic set up of SLA compliance monitoring for the domain they are applied to, e.g., Web services. While reading metrics at specific points of a distributed cluster could be described in some representations such as WSLA there was no prescribed way to address the heterogeneous instrumentation that could be found in today's Cloud scenarios. It was not really necessary because the systems instrumented were assumed to be Web servers, Grid schedulers and so forth.

\cite{baset} provides a survey on current service level provisioning practices and compares the SLAs of five public cloud service providers. It highlights that so far service providers have not adopted systematic methods to detect and alert their customers on service level violations. On the contrary, current cloud service customers need to verify service level violations on their own. Additionally, Baset elaborates on the definition on service level guarantees for cloud services, which are static to each provider and not custom to the clients' needs.

Further research approaches have been proposed to represent SLAs for Cloud deployments. In \cite{kouki} Kouki et al propose the Cloud Service Level Agreement language (CSLA) that enables the definition of SLAs for any type of cloud service. CSLA is intended for dynamic service provisioning environments and is based on the Open Cloud Computing Interface (OCCI) \cite{occi} and the Cloud Computing Reference Architecture of the National Institute of Standards and Technology (NIST) \cite{nist}.


