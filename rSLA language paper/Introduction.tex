\section{Introduction}\label{sec:introduction}

rSLA is a domain specific language (DSL) for expressing and managing service level agreements (SLAs) in a cloud environment. rSLA is coded in Ruby \cite{ruby}, a dynamic language that enables rapid prototyping and application development. 

The rSLA DSL is described by an alphabet and by production rules that help to extend the language. The rSLA programming library provides a runtime engine for deploying and running an rSLA service in a cloud environment.

Although the scientific literature provides plentiful results on automated management of SLAs for distributed computing \cite{wsla, wsag}, cloud markets hesitate to adopt such solutions. Provisioning of cloud services is handled either manually or with software tools that do not embrace cloud service characteristics.

Cloud service management does not yet support automated and transparent solutions for the management of leased resources. Additionally, there is no established standard yet for the automatic expression and management of SLAs for cloud services.

rSLA provides a DSL library for the definition of rSLA objects and a runtime engine to create and process such objects. The DSL enables the automated generation of customized SLAs and the transparent management of cloud service compliance.

The rSLA is deployed on the IBM Bluemix platform \cite{bluemix} as a ruby web service using the sinatra gem\footnote{Sinatra, \url{http://www.sinatrarb.com/}}. A pilot version of the language is currently running for an IBM financial client. The monthly results from using the rSLA language to evaluate the service level compliance of resources leased by the client, showcase the rSLA DSL adequacy in managing cloud services.

How is the paper structured

-what is the problem that the language solves, motivation to solve this problem
-language structure, alphabet, production rules
-language runtime
-current testing, future testing