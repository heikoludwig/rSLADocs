
\section{Problem definition/ Motivation}\label{problem}
\todoheiko{Introduction, Motivation, Abstract}\\
\todogabriel{Introduction, Motivation, Abstract}
Cloud service management has not yet integrated automated and transparent solutions for controlling the provisioning of resources. Cloud customers pay for applications, however they do not have tools to verify their applications' service level values. Similarly, cloud providers do not have tools to evaluate on-demand their service level compliance and to optimally control their resource distribution.

There is no established standard for the automated expression and management of SLAs for cloud services. The research community has proposed solutions for defining SLA context \cite{wsla, wsag} and for processing such information \cite{lisa, lessons} in a distributed computing environment. The cloud industry, however, has not adopted any methods for the automatic management and evaluation of service level values.

SLA terms enclose data values that are retrieved from monitoring. Monitoring in this context means the systematic observation of metric values that are used by one or more services. An SLA may include numerous such values, which in turn are processed for the evaluation of service level objectives (SLOs). In service level management, an SLO verifies if SLA conditions are violated or not. A cloud provider needs to control during service runtime the value levels of countless SLOs from multiple customers. Cloud markets do not yet provide such tools.

A scheduler is used to coordinate the execution of involved processes for the systematic control of service level values. A cloud scheduler executes tasks that define the monitoring of service metrics and that perform the evaluation of service values at defined points during a service runtime. In a cloud environment, a scheduler additionally takes into account the availability of cloud resources and their distribution among service customers. 
 
In service level management, there are functional and operational dependencies between involved entities. For example, a composite metric, as its name implies, represents the composition result from one or more base, as well as composite metric values. The values of composite metrics in an SLA are used for the definition of conditions and objectives on service levels. 

Hence, a schedule configuration for measuring base metric values, may impact the evaluation schedule for one or more composite metric values. Such dependencies raise research questions on how optimal scheduling configurations can apply in a cloud environment for service measurement and evaluation tasks.
 
Evaluation of service levels consists from multiple computing tasks. SLOs are evaluated in scheduled intervals to determine service level compliance. An evaluation process may define a set of SLO conditions.  Such conditions take the form of logical statements that are configured in the SLO definition. Logical statements may require to compare a set of composite values against one or more threshold values. 

rSLA provides an SLA programming library and a service runtime engine for creating and managing SLAs in a cloud environment. The language is not intended to be used only by engineers or ruby developers. An important goal of the rSLA design is to provide a high-level, easy to use and to extend tool that is suitable either for human or machine consumption.


IBM Bluemix is a cloud-based platform to build, run and manage applications. 
