\section{rSLA implementation}\label{runtime} \todomohamed{rSLA implementation}

The rSLA is implemented as a DSL for cloud SLAs. The rSLA programming library, currently, provides support for the following SLA management operations that take place while a provisioned service is running:
\begin{enumerate}
\item SLA creation and activation.
\item monitoring and measurement of service metrics as specified by the SLA context.
\item storage and processing of observed service metric values and of SLO evaluation results. 
\item scheduling of rSLA objects.
\item service level evaluation.
\item notification events: reports.
\end{enumerate}
The next paragraphs highlight rSLA implementation aspects for all supported operations.

\subsection{SLA creation and activation}

As discussed in Chapter \ref{dsl}, rSLA editing takes place using ruby programming blocks. The rSLA language takes advantage of this Ruby coding feature and exposes rSLA objects through multi-lined do..end coding blocks. When an rSLA runtime engine reads a new rSLA block, it generates a new rSLA object that belongs to the block related class. The attributes and function behavior of the generated object are mapped from the ruby block context. 

Listing \ref{basescript} describes an rSLA script for the creation of an SLA and of a base metric. The script can run in an rSLA service runtime environment to generate the two respective objects and to activate a schedule for the value measurement of the base metric.

\begin{minipage}{0.9\textwidth}
\begin{lstlisting}[language=Ruby, basicstyle=\small\normalfont\sffamily, breaklines=true,  captionpos=b, mathescape=true, caption=rSLA SLA (lines 1-4) and basemetric (lines 6-19) creation script, label=basescript, numbers=left, numbersep=5pt, numberstyle=\tiny] 
sla do
  tenant "Demo"
  provider "IBM"
end  

basemetric do
    name "bareMetalProvisioning"
    unit "provisioningtime"
    measurementdirective do
    	entity "baremetal"
    	type "jsonArray"
    	source "http://provisioningxlet.stage1.mybluemix.net/server/baremetal/provisioningtime" 
  	end
  schedule do    
  	frequency "1"
    unit "m"
    method "every"
  end
end
\end{lstlisting}
\end{minipage} 

The block for the creation of the SLA object simply describes two strings that designate the names of the SLA tenant and provider. The rSLA engine reads such strings as the attribute values of the newly created SLA object. Similarly, the rSLA interpreter reads the base metric block and create a new base metric object that inherits the attributes of the rSLA BaseMetric class. 

In the base metric block, a DSL user can define BaseMetric object attributes like the base metric name and measurement unit. Additionally, a DSL user can specify directives for the measurement of the created metric. A measurement directive represents a concept that is inherited from the WSLA specification \cite{wsla}. 

The rSLA DSL exposes a measurement directive as a block of statements that a DSL user can specify. Such statements describe the configuration of a measurement directive object that guides the value measurement of the parent base metric. A measurement directive indicates the result type that is expected from a base metric measurement.

In the measurement directive block, the term entity is used for the representation of restful \footnote{ReST: \url{http://en.wikipedia.org/wiki/Representational_state_transfer}} endpoints. Such may have a URL\footnote{Unique Resource Locator: \url{http://en.wikipedia.org/wiki/Uniform_resource_locator}} form: \url{http://provisioningxlet.stage1.mybluemix.net/server/baremetal/provisioningtime}. A measurement directive object uses an attribute named $source$ to denote the restful endpoint for fetching the relevant base metric value. The rSLA MeasurementDirective class provides an example on how to define measurement directive objects for rSLA base metrics. Such example, like the illustrated block of Listing \ref{basescript} can be extended accordingly. 

Last but not least, a DSL user can specify the details of a schedule for the measurement of the base metric. The schedule block details are explained in Section \ref{schedule}.

\subsection{monitoring and measurement}
Any SLA management framework for cloud services needs a tool for monitoring and measuring service metric values. In a cloud runtime environment, an rSLA service currently uses Java Xlets \cite{xlets} to handle the monitoring of base betric values. Chapter \ref{deployment} describes in detail the configuration and functionality of Xlets with the deployment of rSLA as a compliance service in a cloud environment.

In an rSLA service, monitoring and measurement operations are controlled by one or more schedules. Every base metric follows its own schedule. Measured metric values are collected to a backend database for further processing. 
\subsection{storage and processing}
Currently, rSLA is deployed as a service on the IBM Bluemix platform \cite{bluemix} and is configured to use a Cloudant \cite{cloudant} document database. The values of base metrics are preserved as observation documents with a timestamp on the measurement time and the associated base metric id. Cloudant uses map and reduce statements to collect and to efficiently process data values. 

Observed values of base metrics are stored, while a provisioned service is running. These data values are required for the computation of composite metric values that are used by the service level evaluation. 

rSLA uses CouchRest \cite{couchrest} to access Cloudant through HTTP requests and CouchRest model to map rSLA object properties in the database. Associations between rSLA objects are, at the present time, described through the rSLA Cloudant data schema. Figure \ref{schema} illustrates  associations between rSLA objects in a Cloudant database.

Base, composite metrics as well as SLOs are associated with an SLA instance by a \textit{belongs to} relationship. Similarly, a set of observations \textit{belongs to} a base metric and a set of evaluation and notification data to an SLO. 

The \textit{belongs to} feature is defined in the CouchRest model as a property for the association of documents. It resembles a \textit{has-a} composition relationship \footnote{\url{http://en.wikipedia.org/wiki/Has-a}} where a constituent object belongs to or is part of the definition of another object.

\begin{figure}
\centering
\includegraphics[width=0.8\textwidth]{pics/schema}
\caption{\label{schema} rSLA Cloudant object associations}
\end{figure}

At the current development state of the rSLA language, processing of service metric values may refer to compositions of base and composite metric value sets for the creation of composite metrics or of rSLA expression objects. In rSLA language, the definition of composite metrics represents the outcome from combining a set of base and/or composite metrics. The value computation of such metrics represents a computation that comprises values of other metrics. 

As discussed in Section \ref{dsl}, expressions in rSLA denote free form statements that may take a conditional form or that may define data value computations. The computation for a composite metric value exemplifies an expression in rSLA. Composite metric values are used for the SLO evaluation. rSLA expressions that are defined as either SLO precondition or objectives may refer to composite metric values.

Last but not least, the rSLA language library contains also a time series class that can be used for the application of statistical functions on time-series data sets. 

\subsection{scheduling}\label{schedule}
The rSLA BaseMetric class provides methods for configuring a base metric object with one or more schedules. In an rSLA runtime environment, a scheduler orchestrates the activation, monitoring and processing of rSLA objects.

When the rSLA engine reads an rSLA script file and generates a new SLA object, the SLA object is inactive. The inactive state indicates that the SLA object is not associated with any service metrics that require measurement. It also designates that a scheduler is unaware of the SLA object creation and thus unaware of all rSLA objects that are defined in the new rSLA tree.

On SLA activation, the rSLA engine collects all base metrics that are included under the new SLA and activates them by sending their schedule and id to a scheduler. The scheduler uses the base metric input to create scheduling events for each of the base metrics. 

In rSLA, composite metrics do not include a schedule because the computation of their values depends from SLO evaluation events. The SLO definition specifies a schedule for the evaluation of precondition and objective expressions. Moreover, a scheduler orchestrates notification processes that send evaluation reports on scheduled intervals.

The orchestration of scheduling events represents a topic of on-going work. Currently, rSLA uses the rufus-scheduler that is an in-memory Ruby scheduler. 

\subsection{service level evaluation}

In an rSLA runtime environment, service level evaluation takes place at scheduled intervals. The frequency of service level evaluation is determined by the schedule attributes in the SLO definition. As discussed in Section \ref{dsl}, the definition of a service level objective encloses the description of a precondition and of an objective expressions. These two expressions that follow sequential logic, define the evaluation conditions that designate if the state of an SLO is healthy or not.

We can symbolically define such expressions as the union of composite metric values, hence 


\subsection{notification reports}
can also be alerts
rSLA needs a source for monitoring data and a tool for reporting.